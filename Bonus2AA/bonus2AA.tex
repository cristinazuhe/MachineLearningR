\documentclass{article}
\usepackage[utf8]{inputenc}

\title{Aprendizaje automático: Bonus 2.}
\author{Cristina Zuheros Montes.}
\date{27/04/2016.}

\begin{document}

\maketitle

\section{MATRICES Y OPTIMIZACIÓN.}
 Lagrange propuso una técnica para resolver el siguiente problema de optimización 
 \[
 \begin{array}{c}
 \max_{x,y} g(x,y) \\
 \hbox{Sujeto a } f(x,y)=0
 \end{array}
 \]
 Es decir, buscar el máximo de la función $g$ en un recinto del plano $x-y$ definido por los valores nulos de la función $f$. La solución es transformar este problema de optimización con restricciones en un problema de optimizaciíon sin restricciones y resolver este último derivando e igualando a cero. Para ello construye una nueva función denominada Lagrangiana que se define como
 \[
 \mathcal{L}(x,y,\lambda)=g(x,y)-\lambda f(x,y)
 \]
 siendo $lambda$ una constante y prueba que la solución de óptimo de $\mathcal{L}$ es la misma que la del problema inicial. Por ello para obtener dicha solución solo hay que calcular la solución del sistema de ecuaciones dado por $\nabla_{x,y,\lambda}\mathcal{L}(x,y,\lambda)=0$. En el caso de que exista más de una restricción en igualdad cada una de ellas se añade a la Lagrangiana de la misma manera pero con un $\lambda$ diferente.
 \[
 \mathcal{L}(x,y,\lambda_1,\cdots,\lambda_n)=g(x,y)-\sum_{i=1}^n\lambda_i f_i(x,y)
 \]
 Resolver el siguiente problema: 
     \begin{enumerate}
     	\item La distancia entre dos curvas en el plano está dada por el mínimo de la expresión $\sqrt{(x_1-x_2)^2+(y_1-y_2)^2}$ donde $(x_1,y_1)$ está sobre una de las curvas y $(x_2,y_2)$ está sobre la otra. Calcular la distancia entre la línea $x+y=4$ y la elipse $x^2+2y^2=1$.
     \end{enumerate}
     
 \textbf{Solución}
  
En primer lugar, notamos que la distancia entre la recta $ax+by+c$ y el punto $p=(x,y)$ viene dada por:
\[
dis(x,y):= \frac{|ax+by+c|}{\sqrt{a^{2}+b^{2}}}
\]

En nuestro caso, disponemos de la recta $x+y=4$, luego tendremos:
\[
dis(x,y):= \frac{|x+y-4|}{\sqrt{2}}
\]

Es claro que estamos trabajando en el semiplano definido por la inecuación $x+y-4<0$, luego podemos trabajar con:
 \[
 dis(x,y):= \frac{-x-y+4}{\sqrt{2}} =: g(x,y)
 \]
 
 Para terminar de plantear el problema, sólo falta decir que estaría sujeto a:
  \[
  f(x,y):= x^{2}+2y^{2} = 1
  \]
  
 Ahora ya estamos en disposición de resolver el problema usando los multiplicadores de Lagrange. Para ello defino la siguiente función:
  \[
  \mathcal{L}(x,y,\lambda)=\frac{-x-y+4}{\sqrt{2}}-\lambda (x^{2}+2y^{2}-1)
  \]
  
  Derivamos e igualamos a 0:
\[
  \nabla_{x,y,\lambda}\mathcal{L}(x,y,\lambda)=(\frac{-1}{\sqrt{2}} -2\lambda x,\frac{-1}{\sqrt{2}} -4\lambda y,-x^{2}-2y^{2}+1) =0
\]
Luego tendremos que resolver el sistema:
\[
(1)\ \ \  \frac{-1}{\sqrt{2}} = 2\lambda x  
\]
\[
(2)\ \ \  \frac{-1}{\sqrt{2}} = 4\lambda y    
\]
\[
(3)\ \ \ x^{2}+2y^{2}=1   
\]

De (1) y (2) obtenemos que $x=2y$. Haciendo uso de ello en (3) tenemos $4y^{2}+2y^{2}=1$,
luego $y=\frac{1}{\sqrt{6}}$ o bien $y=-\frac{1}{\sqrt{6}}$.
Por tanto, $x=\sqrt{\frac{2}{3}}$ o bien $x=-\sqrt{\frac{2}{3}}$.\\

Resumiendo, tenemos los siguientes valores para calcular el mínimo y máximo, respectivamente:
\[
(x,y)=(\sqrt{\frac{2}{3}}, \frac{1}{\sqrt{6}})
\]
\[
(x,y)=(-\sqrt{\frac{2}{3}}, -\frac{1}{\sqrt{6}})
\]

Finalmente, calculamos la distancia mínima que vendría dada por:
\[
g(\sqrt{\frac{2}{3}}, \frac{1}{\sqrt{6}}) = \frac{4-\frac{\sqrt{6}}{2}}{\sqrt{2}} = 1.9624
\]

\end{document}
