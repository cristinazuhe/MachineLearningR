\documentclass[]{article}
\usepackage{lmodern}
\usepackage{amssymb,amsmath}
\usepackage{ifxetex,ifluatex}
\usepackage{fixltx2e} % provides \textsubscript
\ifnum 0\ifxetex 1\fi\ifluatex 1\fi=0 % if pdftex
  \usepackage[T1]{fontenc}
  \usepackage[utf8]{inputenc}
\else % if luatex or xelatex
  \ifxetex
    \usepackage{mathspec}
  \else
    \usepackage{fontspec}
  \fi
  \defaultfontfeatures{Ligatures=TeX,Scale=MatchLowercase}
  \newcommand{\euro}{€}
\fi
% use upquote if available, for straight quotes in verbatim environments
\IfFileExists{upquote.sty}{\usepackage{upquote}}{}
% use microtype if available
\IfFileExists{microtype.sty}{%
\usepackage{microtype}
\UseMicrotypeSet[protrusion]{basicmath} % disable protrusion for tt fonts
}{}
\usepackage[margin=1in]{geometry}
\usepackage{hyperref}
\PassOptionsToPackage{usenames,dvipsnames}{color} % color is loaded by hyperref
\hypersetup{unicode=true,
            pdftitle={prac2.R},
            pdfauthor={Cris},
            pdfborder={0 0 0},
            breaklinks=true}
\urlstyle{same}  % don't use monospace font for urls
\usepackage{color}
\usepackage{fancyvrb}
\newcommand{\VerbBar}{|}
\newcommand{\VERB}{\Verb[commandchars=\\\{\}]}
\DefineVerbatimEnvironment{Highlighting}{Verbatim}{commandchars=\\\{\}}
% Add ',fontsize=\small' for more characters per line
\usepackage{framed}
\definecolor{shadecolor}{RGB}{248,248,248}
\newenvironment{Shaded}{\begin{snugshade}}{\end{snugshade}}
\newcommand{\KeywordTok}[1]{\textcolor[rgb]{0.13,0.29,0.53}{\textbf{{#1}}}}
\newcommand{\DataTypeTok}[1]{\textcolor[rgb]{0.13,0.29,0.53}{{#1}}}
\newcommand{\DecValTok}[1]{\textcolor[rgb]{0.00,0.00,0.81}{{#1}}}
\newcommand{\BaseNTok}[1]{\textcolor[rgb]{0.00,0.00,0.81}{{#1}}}
\newcommand{\FloatTok}[1]{\textcolor[rgb]{0.00,0.00,0.81}{{#1}}}
\newcommand{\ConstantTok}[1]{\textcolor[rgb]{0.00,0.00,0.00}{{#1}}}
\newcommand{\CharTok}[1]{\textcolor[rgb]{0.31,0.60,0.02}{{#1}}}
\newcommand{\SpecialCharTok}[1]{\textcolor[rgb]{0.00,0.00,0.00}{{#1}}}
\newcommand{\StringTok}[1]{\textcolor[rgb]{0.31,0.60,0.02}{{#1}}}
\newcommand{\VerbatimStringTok}[1]{\textcolor[rgb]{0.31,0.60,0.02}{{#1}}}
\newcommand{\SpecialStringTok}[1]{\textcolor[rgb]{0.31,0.60,0.02}{{#1}}}
\newcommand{\ImportTok}[1]{{#1}}
\newcommand{\CommentTok}[1]{\textcolor[rgb]{0.56,0.35,0.01}{\textit{{#1}}}}
\newcommand{\DocumentationTok}[1]{\textcolor[rgb]{0.56,0.35,0.01}{\textbf{\textit{{#1}}}}}
\newcommand{\AnnotationTok}[1]{\textcolor[rgb]{0.56,0.35,0.01}{\textbf{\textit{{#1}}}}}
\newcommand{\CommentVarTok}[1]{\textcolor[rgb]{0.56,0.35,0.01}{\textbf{\textit{{#1}}}}}
\newcommand{\OtherTok}[1]{\textcolor[rgb]{0.56,0.35,0.01}{{#1}}}
\newcommand{\FunctionTok}[1]{\textcolor[rgb]{0.00,0.00,0.00}{{#1}}}
\newcommand{\VariableTok}[1]{\textcolor[rgb]{0.00,0.00,0.00}{{#1}}}
\newcommand{\ControlFlowTok}[1]{\textcolor[rgb]{0.13,0.29,0.53}{\textbf{{#1}}}}
\newcommand{\OperatorTok}[1]{\textcolor[rgb]{0.81,0.36,0.00}{\textbf{{#1}}}}
\newcommand{\BuiltInTok}[1]{{#1}}
\newcommand{\ExtensionTok}[1]{{#1}}
\newcommand{\PreprocessorTok}[1]{\textcolor[rgb]{0.56,0.35,0.01}{\textit{{#1}}}}
\newcommand{\AttributeTok}[1]{\textcolor[rgb]{0.77,0.63,0.00}{{#1}}}
\newcommand{\RegionMarkerTok}[1]{{#1}}
\newcommand{\InformationTok}[1]{\textcolor[rgb]{0.56,0.35,0.01}{\textbf{\textit{{#1}}}}}
\newcommand{\WarningTok}[1]{\textcolor[rgb]{0.56,0.35,0.01}{\textbf{\textit{{#1}}}}}
\newcommand{\AlertTok}[1]{\textcolor[rgb]{0.94,0.16,0.16}{{#1}}}
\newcommand{\ErrorTok}[1]{\textcolor[rgb]{0.64,0.00,0.00}{\textbf{{#1}}}}
\newcommand{\NormalTok}[1]{{#1}}
\usepackage{graphicx,grffile}
\makeatletter
\def\maxwidth{\ifdim\Gin@nat@width>\linewidth\linewidth\else\Gin@nat@width\fi}
\def\maxheight{\ifdim\Gin@nat@height>\textheight\textheight\else\Gin@nat@height\fi}
\makeatother
% Scale images if necessary, so that they will not overflow the page
% margins by default, and it is still possible to overwrite the defaults
% using explicit options in \includegraphics[width, height, ...]{}
\setkeys{Gin}{width=\maxwidth,height=\maxheight,keepaspectratio}
\setlength{\parindent}{0pt}
\setlength{\parskip}{6pt plus 2pt minus 1pt}
\setlength{\emergencystretch}{3em}  % prevent overfull lines
\providecommand{\tightlist}{%
  \setlength{\itemsep}{0pt}\setlength{\parskip}{0pt}}
\setcounter{secnumdepth}{0}

%%% Use protect on footnotes to avoid problems with footnotes in titles
\let\rmarkdownfootnote\footnote%
\def\footnote{\protect\rmarkdownfootnote}

%%% Change title format to be more compact
\usepackage{titling}

% Create subtitle command for use in maketitle
\newcommand{\subtitle}[1]{
  \posttitle{
    \begin{center}\large#1\end{center}
    }
}

\setlength{\droptitle}{-2em}
  \title{prac2.R}
  \pretitle{\vspace{\droptitle}\centering\huge}
  \posttitle{\par}
  \author{Cris}
  \preauthor{\centering\large\emph}
  \postauthor{\par}
  \predate{\centering\large\emph}
  \postdate{\par}
  \date{Sat Apr 30 01:50:52 2016}



% Redefines (sub)paragraphs to behave more like sections
\ifx\paragraph\undefined\else
\let\oldparagraph\paragraph
\renewcommand{\paragraph}[1]{\oldparagraph{#1}\mbox{}}
\fi
\ifx\subparagraph\undefined\else
\let\oldsubparagraph\subparagraph
\renewcommand{\subparagraph}[1]{\oldsubparagraph{#1}\mbox{}}
\fi

\begin{document}
\maketitle

\begin{Shaded}
\begin{Highlighting}[]
\NormalTok{#######################################################################}
\NormalTok{################## Cristina Zuheros Montes - 2016 #####################}
\NormalTok{#######################################################################}

\CommentTok{#Funcion auxiliar para ir parado la ejecucion}
\NormalTok{pulsaTecla <-}\StringTok{ }\NormalTok{function()\{}
  \KeywordTok{cat} \NormalTok{(}\StringTok{"Pulse Intro para continuar..."}\NormalTok{)}
  \NormalTok{line <-}\StringTok{ }\KeywordTok{readline}\NormalTok{()}
\NormalTok{\}}

\NormalTok{pintar_grafica =}\StringTok{ }\NormalTok{function(f) \{}
  \NormalTok{x=y=}\KeywordTok{seq}\NormalTok{(-}\DecValTok{50}\NormalTok{,}\DecValTok{50}\NormalTok{,}\DataTypeTok{by=}\FloatTok{0.1}\NormalTok{)}
  \NormalTok{z =}\StringTok{ }\KeywordTok{outer}\NormalTok{(x,y,}\DataTypeTok{FUN=}\NormalTok{f)}
  \KeywordTok{contour}\NormalTok{(x,y,z, }\DataTypeTok{levels=}\DecValTok{0}\NormalTok{:}\DecValTok{3}\NormalTok{, }\DataTypeTok{drawlabels =} \OtherTok{TRUE}\NormalTok{,}\DataTypeTok{add =} \OtherTok{TRUE}\NormalTok{, }\DataTypeTok{col=}\StringTok{"purple"}\NormalTok{)}
\NormalTok{\}}

\NormalTok{simula_unifM =}\StringTok{ }\NormalTok{function (}\DataTypeTok{N=}\DecValTok{2}\NormalTok{,}\DataTypeTok{dims=}\DecValTok{2}\NormalTok{, }\DataTypeTok{rango =} \KeywordTok{c}\NormalTok{(}\DecValTok{0}\NormalTok{,}\DecValTok{1}\NormalTok{))\{}
  \NormalTok{m =}\StringTok{ }\KeywordTok{matrix}\NormalTok{(}\KeywordTok{runif}\NormalTok{(N*dims, }\DataTypeTok{min=}\NormalTok{rango[}\DecValTok{1}\NormalTok{], }\DataTypeTok{max=}\NormalTok{rango[}\DecValTok{2}\NormalTok{]),}
             \DataTypeTok{nrow =} \NormalTok{N, }\DataTypeTok{ncol=}\NormalTok{dims, }\DataTypeTok{byrow=}\NormalTok{T)}
\NormalTok{\}}

\NormalTok{simula_recta =}\StringTok{ }\NormalTok{function (}\DataTypeTok{intervalo =} \KeywordTok{c}\NormalTok{(-}\DecValTok{1}\NormalTok{,}\DecValTok{1}\NormalTok{),}\DataTypeTok{visible=}\NormalTok{F, }\DataTypeTok{ptos =} \OtherTok{NULL}\NormalTok{)\{}
  \NormalTok{if(}\KeywordTok{is.null}\NormalTok{(ptos)) m =}\StringTok{ }\KeywordTok{simula_unifM}\NormalTok{(}\DecValTok{2}\NormalTok{,}\DecValTok{2}\NormalTok{,intervalo)}
  \NormalTok{a =}\StringTok{ }\NormalTok{(m[}\DecValTok{1}\NormalTok{,}\DecValTok{2}\NormalTok{] -}\StringTok{ }\NormalTok{m[}\DecValTok{2}\NormalTok{,}\DecValTok{2}\NormalTok{]) /}\StringTok{ }\NormalTok{(m[}\DecValTok{1}\NormalTok{,}\DecValTok{1}\NormalTok{]-m[}\DecValTok{2}\NormalTok{,}\DecValTok{1}\NormalTok{]) }\CommentTok{# calculo de la pendiente}
  \NormalTok{b =}\StringTok{ }\NormalTok{m[}\DecValTok{1}\NormalTok{,}\DecValTok{2}\NormalTok{]-a*m[}\DecValTok{1}\NormalTok{,}\DecValTok{1}\NormalTok{]}
  \NormalTok{if (visible) \{}
  \CommentTok{#  if (dev.cur()==1) # no esta abierto el dispositivo lo abre con plot}
    \KeywordTok{plot}\NormalTok{(}\OtherTok{NULL}\NormalTok{, }\OtherTok{NULL}\NormalTok{, }\DataTypeTok{type=}\StringTok{"n"}\NormalTok{, }\DataTypeTok{xlim=}\NormalTok{intervalo, }\DataTypeTok{ylim=}\NormalTok{intervalo)}
    \KeywordTok{abline}\NormalTok{(b,a)}
    \KeywordTok{points}\NormalTok{(m,}\DataTypeTok{col=}\DecValTok{3}\NormalTok{) }\CommentTok{#pinta en verde los puntos}
  \NormalTok{\}}
  \KeywordTok{return}\NormalTok{(}\KeywordTok{c}\NormalTok{(a,b))}
\NormalTok{\}}

\NormalTok{pinta_puntos =}\StringTok{ }\NormalTok{function(m,}\DataTypeTok{intervalo =} \KeywordTok{c}\NormalTok{(-}\DecValTok{1}\NormalTok{,}\DecValTok{1}\NormalTok{) ,}\DataTypeTok{etiqueta=}\DecValTok{1}\NormalTok{)\{}
  \NormalTok{nptos=}\KeywordTok{nrow}\NormalTok{(m)}
  \KeywordTok{plot}\NormalTok{(m,}\DataTypeTok{xlim=}\NormalTok{intervalo, }\DataTypeTok{ylim =} \NormalTok{intervalo, }\DataTypeTok{xlab=}\KeywordTok{paste}\NormalTok{(}\StringTok{"Pinta "}\NormalTok{,nptos,}\StringTok{" Puntos"}\NormalTok{), }\DataTypeTok{ylab=}\StringTok{""}\NormalTok{,}\DataTypeTok{col=}\NormalTok{etiqueta}\DecValTok{+3}\NormalTok{)}
\NormalTok{\}}
\NormalTok{########################################################################}
\NormalTok{############################### SECCION 1 ##############################}
\NormalTok{########################################################################}
\KeywordTok{print}\NormalTok{(}\StringTok{"###############################SECCION 1###################################"}\NormalTok{)}
\end{Highlighting}
\end{Shaded}

\begin{verbatim}
## [1] "###############################SECCION 1###################################"
\end{verbatim}

\begin{Shaded}
\begin{Highlighting}[]
\NormalTok{###################################################}
\NormalTok{###################EJERCICIO 1.1###################}
\NormalTok{###################################################}
\KeywordTok{print}\NormalTok{(}\StringTok{"###############################Ejercicio 1###################################"}\NormalTok{)}
\end{Highlighting}
\end{Shaded}

\begin{verbatim}
## [1] "###############################Ejercicio 1###################################"
\end{verbatim}

\begin{Shaded}
\begin{Highlighting}[]
\NormalTok{fs1e1f1 =}\StringTok{ }\NormalTok{function(x,y) ((x*}\KeywordTok{exp}\NormalTok{(y))-}\StringTok{ }\NormalTok{(}\DecValTok{2}\NormalTok{*y*}\KeywordTok{exp}\NormalTok{(-x)))^}\DecValTok{2}  \CommentTok{#defino la funcion}
\NormalTok{fs1e1f2 =}\StringTok{ }\NormalTok{function(x,y) x^}\DecValTok{2} \NormalTok{+}\StringTok{  }\DecValTok{2}\NormalTok{*y^}\DecValTok{2} \NormalTok{+}\DecValTok{2}\NormalTok{*}\KeywordTok{sin}\NormalTok{(}\DecValTok{2}\NormalTok{*pi*x)*}\KeywordTok{sin}\NormalTok{(}\DecValTok{2}\NormalTok{*pi*y)  }\CommentTok{#defino la funcion}

\NormalTok{graddesc =}\StringTok{ }\NormalTok{function(}
  \DataTypeTok{FUN =} \NormalTok{function(x, y) ((x*}\KeywordTok{exp}\NormalTok{(y))-}\StringTok{ }\NormalTok{(}\DecValTok{2}\NormalTok{*y*}\KeywordTok{exp}\NormalTok{(-x)))^}\DecValTok{2}\NormalTok{, }\DataTypeTok{interx =} \KeywordTok{c}\NormalTok{(-}\FloatTok{1.25}\NormalTok{, }\FloatTok{1.25}\NormalTok{), }\DataTypeTok{intery=}\KeywordTok{c}\NormalTok{(-}\FloatTok{1.25}\NormalTok{,}\FloatTok{1.25}\NormalTok{),}
  \DataTypeTok{val_ini=}\KeywordTok{c}\NormalTok{(}\DecValTok{1}\NormalTok{,}\DecValTok{1}\NormalTok{), }\DataTypeTok{tasa=}\FloatTok{0.01}\NormalTok{, }\DataTypeTok{tope =} \DecValTok{10}\NormalTok{^(-}\DecValTok{14}\NormalTok{), }\DataTypeTok{mimain=}\StringTok{""}\NormalTok{, }\DataTypeTok{maxiter=}\DecValTok{200}\NormalTok{)\{}
    \KeywordTok{plot}\NormalTok{(}\OtherTok{NULL}\NormalTok{,}\OtherTok{NULL}\NormalTok{, }\DataTypeTok{xlim =} \NormalTok{interx, }\DataTypeTok{ylim=}\NormalTok{intery, }\DataTypeTok{xlab=}\StringTok{"x"}\NormalTok{, }\DataTypeTok{ylab=}\StringTok{"y"}\NormalTok{, }\DataTypeTok{main =} \NormalTok{mimain)}
    \KeywordTok{pintar_grafica}\NormalTok{(FUN)}
    \NormalTok{coste_funcion=}\KeywordTok{FUN}\NormalTok{(val_ini[}\DecValTok{1}\NormalTok{], val_ini[}\DecValTok{2}\NormalTok{])}
    \NormalTok{contador=}\DecValTok{0}
    \NormalTok{while(coste_funcion>tope &&}\StringTok{ }\NormalTok{contador<maxiter)\{}
      \NormalTok{dxy_fs1e1 =}\StringTok{ }\KeywordTok{deriv}\NormalTok{(}\KeywordTok{as.expression}\NormalTok{(}\KeywordTok{body}\NormalTok{(FUN)), }\KeywordTok{c}\NormalTok{(}\StringTok{"x"}\NormalTok{,}\StringTok{"y"}\NormalTok{), }\DataTypeTok{function.arg=}\OtherTok{TRUE}\NormalTok{)}
      \NormalTok{val_sig =}\StringTok{ }\NormalTok{val_ini -}\StringTok{ }\NormalTok{tasa*}\KeywordTok{attr}\NormalTok{(}\KeywordTok{dxy_fs1e1}\NormalTok{(val_ini[}\DecValTok{1}\NormalTok{],val_ini[}\DecValTok{2}\NormalTok{]), }\StringTok{'gradient'}\NormalTok{)}
      \NormalTok{coste_funcion =}\StringTok{ }\KeywordTok{FUN}\NormalTok{(val_sig[}\DecValTok{1}\NormalTok{], val_sig[}\DecValTok{2}\NormalTok{])}
      \NormalTok{val_ini =}\StringTok{ }\NormalTok{val_sig}
      \NormalTok{contador=contador}\DecValTok{+1}
      \KeywordTok{points}\NormalTok{(val_sig, }\DataTypeTok{col=}\StringTok{"orange"}\NormalTok{)}
    \NormalTok{\}}
    \KeywordTok{points}\NormalTok{(val_sig, }\DataTypeTok{col=}\StringTok{"red"}\NormalTok{)}
    \KeywordTok{print}\NormalTok{(}\StringTok{"Numero de iteraciones realizadas:"}\NormalTok{)}
    \KeywordTok{print}\NormalTok{(contador)}
    \KeywordTok{print}\NormalTok{(}\StringTok{"Valor obtenido:"}\NormalTok{)}
    \KeywordTok{print}\NormalTok{(val_ini)}
\NormalTok{\}}
\KeywordTok{print}\NormalTok{(}\StringTok{"*************************Primera funcion*************************"}\NormalTok{)}
\end{Highlighting}
\end{Shaded}

\begin{verbatim}
## [1] "*************************Primera funcion*************************"
\end{verbatim}

\begin{Shaded}
\begin{Highlighting}[]
\CommentTok{#graddesc(mimain="1.1.a)Primera funcion", interx = c(-2,2), intery=c(-2,2), tasa=0.1)}
\KeywordTok{print}\NormalTok{(}\StringTok{"*************************Segunda funcion*************************"}\NormalTok{)}
\end{Highlighting}
\end{Shaded}

\begin{verbatim}
## [1] "*************************Segunda funcion*************************"
\end{verbatim}

\begin{Shaded}
\begin{Highlighting}[]
\KeywordTok{print}\NormalTok{(}\StringTok{"*************************Inicial (1,1)*************************"}\NormalTok{)}
\end{Highlighting}
\end{Shaded}

\begin{verbatim}
## [1] "*************************Inicial (1,1)*************************"
\end{verbatim}

\begin{Shaded}
\begin{Highlighting}[]
\CommentTok{#graddesc(FUN = fs1e1f2,mimain="1.1.b)Segunda funcion (1,1)", maxiter=50)}

\KeywordTok{print}\NormalTok{(}\StringTok{"*************************Inicial (-1,-1)***********************"}\NormalTok{)}
\end{Highlighting}
\end{Shaded}

\begin{verbatim}
## [1] "*************************Inicial (-1,-1)***********************"
\end{verbatim}

\begin{Shaded}
\begin{Highlighting}[]
\CommentTok{#graddesc(FUN = fs1e1f2, val_ini=c(-1,-1),mimain="1.1.b)Segunda funcion (-1,-1)", maxiter=50)}

\KeywordTok{print}\NormalTok{(}\StringTok{"***********************Inicial (0.1,0.1)************************"}\NormalTok{)}
\end{Highlighting}
\end{Shaded}

\begin{verbatim}
## [1] "***********************Inicial (0.1,0.1)************************"
\end{verbatim}

\begin{Shaded}
\begin{Highlighting}[]
\CommentTok{#graddesc(FUN = fs1e1f2, val_ini=c(0.1,0.1),mimain="1.1.b)Segunda funcion (0.1,0.1)", maxiter=50)}

\KeywordTok{print}\NormalTok{(}\StringTok{"**********************Inicial (-0.5,-0.5)***********************"}\NormalTok{)}
\end{Highlighting}
\end{Shaded}

\begin{verbatim}
## [1] "**********************Inicial (-0.5,-0.5)***********************"
\end{verbatim}

\begin{Shaded}
\begin{Highlighting}[]
\CommentTok{#graddesc(FUN = fs1e1f2, val_ini=c(-0.5,-0.5),mimain="1.1.b)Segunda funcion (-0.5,-0.5)", maxiter=50)}

\NormalTok{###################################################}
\NormalTok{###################EJERCICIO 1.2###################}
\NormalTok{###################################################}
\KeywordTok{print}\NormalTok{(}\StringTok{"###############################Ejercicio 2###################################"}\NormalTok{)}
\end{Highlighting}
\end{Shaded}

\begin{verbatim}
## [1] "###############################Ejercicio 2###################################"
\end{verbatim}

\begin{Shaded}
\begin{Highlighting}[]
\NormalTok{coorddesc =}\StringTok{ }\NormalTok{function(}
  \DataTypeTok{FUN =} \NormalTok{function(x, y) ((x*}\KeywordTok{exp}\NormalTok{(y))-}\StringTok{ }\NormalTok{(}\DecValTok{2}\NormalTok{*y*}\KeywordTok{exp}\NormalTok{(-x)))^}\DecValTok{2}\NormalTok{, }\DataTypeTok{interx =} \KeywordTok{c}\NormalTok{(-}\DecValTok{3}\NormalTok{, }\DecValTok{3}\NormalTok{), }\DataTypeTok{intery=}\KeywordTok{c}\NormalTok{(-}\DecValTok{3}\NormalTok{,}\DecValTok{3}\NormalTok{),}
  \DataTypeTok{val_ini=}\KeywordTok{c}\NormalTok{(}\DecValTok{1}\NormalTok{,}\DecValTok{1}\NormalTok{), }\DataTypeTok{tasa=}\FloatTok{0.1}\NormalTok{, }\DataTypeTok{tope =} \DecValTok{10}\NormalTok{^(-}\DecValTok{14}\NormalTok{), }\DataTypeTok{mimain=}\StringTok{""}\NormalTok{, }\DataTypeTok{maxiter=}\DecValTok{15}\NormalTok{)\{}
  \KeywordTok{plot}\NormalTok{(}\OtherTok{NULL}\NormalTok{,}\OtherTok{NULL}\NormalTok{, }\DataTypeTok{xlim =} \NormalTok{interx, }\DataTypeTok{ylim=}\NormalTok{intery, }\DataTypeTok{xlab=}\StringTok{"x"}\NormalTok{, }\DataTypeTok{ylab=}\StringTok{"y"}\NormalTok{, }\DataTypeTok{main =} \NormalTok{mimain)}
  \KeywordTok{pintar_grafica}\NormalTok{(FUN)}
  \NormalTok{coste_funcion=}\KeywordTok{FUN}\NormalTok{(val_ini[}\DecValTok{1}\NormalTok{], val_ini[}\DecValTok{2}\NormalTok{])}
  \NormalTok{contador=}\DecValTok{0}
  \NormalTok{val_sig =}\StringTok{ }\NormalTok{val_ini}
  \NormalTok{while(coste_funcion>tope &&}\StringTok{ }\NormalTok{contador<maxiter)\{}
    \NormalTok{dxy_fs1e1 =}\StringTok{ }\KeywordTok{deriv}\NormalTok{(}\KeywordTok{as.expression}\NormalTok{(}\KeywordTok{body}\NormalTok{(FUN)), }\KeywordTok{c}\NormalTok{(}\StringTok{"x"}\NormalTok{,}\StringTok{"y"}\NormalTok{), }\DataTypeTok{function.arg=}\OtherTok{TRUE}\NormalTok{)}
    \NormalTok{val_sig[}\DecValTok{1}\NormalTok{] =}\StringTok{ }\NormalTok{val_ini[}\DecValTok{1}\NormalTok{] -}\StringTok{ }\NormalTok{tasa*(}\KeywordTok{attr}\NormalTok{(}\KeywordTok{dxy_fs1e1}\NormalTok{(val_ini[}\DecValTok{1}\NormalTok{],val_ini[}\DecValTok{2}\NormalTok{]), }\StringTok{'gradient'}\NormalTok{))[}\DecValTok{1}\NormalTok{]}
    \NormalTok{val_sig[}\DecValTok{2}\NormalTok{] =}\StringTok{ }\NormalTok{val_ini[}\DecValTok{2}\NormalTok{] -}\StringTok{ }\NormalTok{tasa*(}\KeywordTok{attr}\NormalTok{(}\KeywordTok{dxy_fs1e1}\NormalTok{(val_sig[}\DecValTok{1}\NormalTok{],val_ini[}\DecValTok{2}\NormalTok{]), }\StringTok{'gradient'}\NormalTok{))[}\DecValTok{2}\NormalTok{]}
    \NormalTok{coste_funcion =}\StringTok{ }\KeywordTok{FUN}\NormalTok{(val_sig[}\DecValTok{1}\NormalTok{], val_sig[}\DecValTok{2}\NormalTok{])}
    \NormalTok{val_ini =}\StringTok{ }\NormalTok{val_sig}
    \NormalTok{contador=contador}\DecValTok{+1}
    \KeywordTok{points}\NormalTok{(val_sig, }\DataTypeTok{col=}\StringTok{"orange"}\NormalTok{)}
  \NormalTok{\}}
  \KeywordTok{points}\NormalTok{(val_sig, }\DataTypeTok{col=}\StringTok{"red"}\NormalTok{)}
  \KeywordTok{print}\NormalTok{(}\StringTok{"Numero de iteraciones realizadas:"}\NormalTok{)}
  \KeywordTok{print}\NormalTok{(contador)}
  \KeywordTok{print}\NormalTok{(}\StringTok{"Valor obtenido:"}\NormalTok{)}
  \KeywordTok{print}\NormalTok{(val_ini)}
\NormalTok{\}}

\KeywordTok{print}\NormalTok{(}\StringTok{"*************************15 Iteraciones*************************"}\NormalTok{)}
\end{Highlighting}
\end{Shaded}

\begin{verbatim}
## [1] "*************************15 Iteraciones*************************"
\end{verbatim}

\begin{Shaded}
\begin{Highlighting}[]
\CommentTok{#coorddesc(mimain="1.2. 15 Iteraciones.")}
\KeywordTok{print}\NormalTok{(}\StringTok{"*************************30 Iteraciones*************************"}\NormalTok{)}
\end{Highlighting}
\end{Shaded}

\begin{verbatim}
## [1] "*************************30 Iteraciones*************************"
\end{verbatim}

\begin{Shaded}
\begin{Highlighting}[]
\CommentTok{#coorddesc(maxiter=30, mimain="1.2. 30 Iteraciones.")}

\NormalTok{###################################################}
\NormalTok{###################EJERCICIO 1.3###################}
\NormalTok{###################################################}
\KeywordTok{print}\NormalTok{(}\StringTok{"###############################Ejercicio 3###################################"}\NormalTok{)}
\end{Highlighting}
\end{Shaded}

\begin{verbatim}
## [1] "###############################Ejercicio 3###################################"
\end{verbatim}

\begin{Shaded}
\begin{Highlighting}[]
\NormalTok{newton =}\StringTok{ }\NormalTok{function(}
  \DataTypeTok{FUN =} \NormalTok{function(x, y) x^}\DecValTok{2} \NormalTok{+}\StringTok{  }\DecValTok{2}\NormalTok{*y^}\DecValTok{2} \NormalTok{+}\DecValTok{2}\NormalTok{*}\KeywordTok{sin}\NormalTok{(}\DecValTok{2}\NormalTok{*pi*x)*}\KeywordTok{sin}\NormalTok{(}\DecValTok{2}\NormalTok{*pi*y), }\DataTypeTok{interx =} \KeywordTok{c}\NormalTok{(-}\DecValTok{3}\NormalTok{, }\DecValTok{3}\NormalTok{), }\DataTypeTok{intery=}\KeywordTok{c}\NormalTok{(-}\DecValTok{3}\NormalTok{,}\DecValTok{3}\NormalTok{),}
  \DataTypeTok{val_ini=}\KeywordTok{as.matrix}\NormalTok{(}\KeywordTok{rbind}\NormalTok{(}\DecValTok{1}\NormalTok{,}\DecValTok{1}\NormalTok{)), }\DataTypeTok{tasa=}\FloatTok{0.1}\NormalTok{, }\DataTypeTok{tope =} \FloatTok{0.00001}\NormalTok{, }\DataTypeTok{mimain=}\StringTok{""}\NormalTok{, }\DataTypeTok{maxiter=}\DecValTok{15}\NormalTok{)\{}

  \KeywordTok{plot}\NormalTok{(}\OtherTok{NULL}\NormalTok{,}\OtherTok{NULL}\NormalTok{, }\DataTypeTok{xlim =} \NormalTok{interx, }\DataTypeTok{ylim=}\NormalTok{intery, }\DataTypeTok{xlab=}\StringTok{"x"}\NormalTok{, }\DataTypeTok{ylab=}\StringTok{"y"}\NormalTok{, }\DataTypeTok{main =} \NormalTok{mimain)}
  \KeywordTok{pintar_grafica}\NormalTok{(FUN)}
  \NormalTok{val_ini =}\StringTok{ }\KeywordTok{as.matrix}\NormalTok{(}\KeywordTok{rbind}\NormalTok{(}\DecValTok{1}\NormalTok{,}\DecValTok{1}\NormalTok{)) }
  \NormalTok{coste_funcion=}\KeywordTok{FUN}\NormalTok{(val_ini[}\DecValTok{1}\NormalTok{], val_ini[}\DecValTok{2}\NormalTok{])}
  \NormalTok{contador=}\DecValTok{0}
  \NormalTok{val_sig =}\StringTok{ }\NormalTok{val_ini}
  \NormalTok{diferencia_coste =}\StringTok{ }\DecValTok{200}
  \NormalTok{while(diferencia_coste>tope &&}\StringTok{ }\NormalTok{contador<maxiter)\{}
    \NormalTok{hess_fs1e1f1 =}\StringTok{ }\KeywordTok{deriv}\NormalTok{(}\KeywordTok{as.expression}\NormalTok{(}\KeywordTok{body}\NormalTok{(FUN)), }\KeywordTok{c}\NormalTok{(}\StringTok{"x"}\NormalTok{,}\StringTok{"y"}\NormalTok{), }\DataTypeTok{hessian=}\OtherTok{TRUE}\NormalTok{, }\DataTypeTok{function.arg=}\OtherTok{TRUE}\NormalTok{)}
    \NormalTok{hessiana_fs1e1f1 =}\StringTok{ }\KeywordTok{as.matrix}\NormalTok{(}\KeywordTok{rbind}\NormalTok{(}\KeywordTok{c}\NormalTok{((}\KeywordTok{attr}\NormalTok{(}\KeywordTok{hess_fs1e1f1}\NormalTok{(val_ini[}\DecValTok{1}\NormalTok{],val_ini[}\DecValTok{2}\NormalTok{]), }\StringTok{'hessian'}\NormalTok{))[}\DecValTok{1}\NormalTok{],}
                                         \NormalTok{(}\KeywordTok{attr}\NormalTok{(}\KeywordTok{hess_fs1e1f1}\NormalTok{(val_ini[}\DecValTok{1}\NormalTok{],val_ini[}\DecValTok{2}\NormalTok{]), }\StringTok{'hessian'}\NormalTok{))[}\DecValTok{2}\NormalTok{]),}
                                       \KeywordTok{c}\NormalTok{((}\KeywordTok{attr}\NormalTok{(}\KeywordTok{hess_fs1e1f1}\NormalTok{(val_ini[}\DecValTok{1}\NormalTok{],val_ini[}\DecValTok{2}\NormalTok{]), }\StringTok{'hessian'}\NormalTok{))[}\DecValTok{3}\NormalTok{],}
                                         \NormalTok{(}\KeywordTok{attr}\NormalTok{(}\KeywordTok{hess_fs1e1f1}\NormalTok{(val_ini[}\DecValTok{1}\NormalTok{],val_ini[}\DecValTok{2}\NormalTok{]), }\StringTok{'hessian'}\NormalTok{))[}\DecValTok{4}\NormalTok{])))}
    \NormalTok{gradiente_fs1e1f1 =}\StringTok{ }\KeywordTok{as.matrix}\NormalTok{(}\KeywordTok{rbind}\NormalTok{((}\KeywordTok{attr}\NormalTok{(}\KeywordTok{hess_fs1e1f1}\NormalTok{(val_ini[}\DecValTok{1}\NormalTok{],val_ini[}\DecValTok{2}\NormalTok{]), }\StringTok{'grad'}\NormalTok{))[}\DecValTok{1}\NormalTok{],}
                                        \NormalTok{(}\KeywordTok{attr}\NormalTok{(}\KeywordTok{hess_fs1e1f1}\NormalTok{(val_ini[}\DecValTok{1}\NormalTok{],val_ini[}\DecValTok{2}\NormalTok{]), }\StringTok{'grad'}\NormalTok{))[}\DecValTok{2}\NormalTok{]))}
    \NormalTok{hessiana_fs1e1f1 =}\StringTok{ }\KeywordTok{solve}\NormalTok{(hessiana_fs1e1f1)}

    \NormalTok{val_sig =}\StringTok{ }\NormalTok{val_ini -hessiana_fs1e1f1%*%gradiente_fs1e1f1}
    \NormalTok{diferencia_coste =}\StringTok{ }\KeywordTok{abs}\NormalTok{(}\KeywordTok{FUN}\NormalTok{(val_sig[}\DecValTok{1}\NormalTok{], val_sig[}\DecValTok{2}\NormalTok{]) -}\StringTok{ }\KeywordTok{FUN}\NormalTok{(val_ini[}\DecValTok{1}\NormalTok{], val_ini[}\DecValTok{2}\NormalTok{]))}
    \NormalTok{val_ini =}\StringTok{ }\NormalTok{val_sig}
    \NormalTok{contador=contador}\DecValTok{+1}
    \KeywordTok{points}\NormalTok{(val_sig, }\DataTypeTok{col=}\StringTok{"orange"}\NormalTok{)}
    \KeywordTok{print}\NormalTok{(}\KeywordTok{FUN}\NormalTok{(val_sig[}\DecValTok{1}\NormalTok{], val_sig[}\DecValTok{2}\NormalTok{]))}
  \NormalTok{\}}
  \KeywordTok{points}\NormalTok{(val_sig, }\DataTypeTok{col=}\StringTok{"red"}\NormalTok{)}
  \KeywordTok{print}\NormalTok{(}\StringTok{"Numero de iteraciones realizadas:"}\NormalTok{)}
  \KeywordTok{print}\NormalTok{(contador)}
  \KeywordTok{print}\NormalTok{(}\StringTok{"Valor obtenido:"}\NormalTok{)}
  \KeywordTok{print}\NormalTok{(val_sig)}
\NormalTok{\}}

\CommentTok{#newton()}

\NormalTok{###################################################}
\NormalTok{###################EJERCICIO 1.4###################}
\NormalTok{###################################################}
\KeywordTok{print}\NormalTok{(}\StringTok{"###############################Ejercicio 4###################################"}\NormalTok{)}
\end{Highlighting}
\end{Shaded}

\begin{verbatim}
## [1] "###############################Ejercicio 4###################################"
\end{verbatim}

\begin{Shaded}
\begin{Highlighting}[]
\NormalTok{N=}\DecValTok{100}
\NormalTok{rango =}\StringTok{ }\KeywordTok{c}\NormalTok{(-}\DecValTok{1}\NormalTok{,}\DecValTok{1}\NormalTok{)}
\NormalTok{datos =}\StringTok{ }\KeywordTok{simula_unifM} \NormalTok{(N,}\DecValTok{2}\NormalTok{,rango)}
\NormalTok{coef =}\StringTok{ }\KeywordTok{simula_recta}\NormalTok{(rango) }
\NormalTok{f4 =}\StringTok{ }\NormalTok{function(pto,coef) \{          }\CommentTok{#funcion de la recta}
  \NormalTok{pto[}\DecValTok{2}\NormalTok{]-pto[}\DecValTok{1}\NormalTok{]*coef[}\DecValTok{1}\NormalTok{]-coef[}\DecValTok{2}\NormalTok{]}
\NormalTok{\}}
\NormalTok{z0 =}\StringTok{ }\KeywordTok{apply}\NormalTok{(datos,}\DecValTok{1}\NormalTok{,f4,coef)        }\CommentTok{#obtiene los valores de la funcion para datos}
\NormalTok{etiqueta =}\StringTok{ }\KeywordTok{sign}\NormalTok{(z0)                }\CommentTok{# apartir de ellos crea las etiquetas}
\CommentTok{#pinta_puntos(datos, intervalo = rango,etiqueta=etiqueta) # utiliza las etiquetas }
\CommentTok{#abline(coef[2],coef[1])            # pinta la recta}


\NormalTok{tasa =}\StringTok{ }\FloatTok{0.01}
\NormalTok{vector_w_ini=}\KeywordTok{c}\NormalTok{(}\DecValTok{0}\NormalTok{,}\DecValTok{0}\NormalTok{) }

\NormalTok{reglog =}\StringTok{ }\NormalTok{function(datos,etiqueta,}\DataTypeTok{vector_w_ini=}\KeywordTok{c}\NormalTok{(}\DecValTok{0}\NormalTok{,}\DecValTok{0}\NormalTok{), }\DataTypeTok{tasa=}\FloatTok{0.01}\NormalTok{)\{}
  \NormalTok{N=}\StringTok{ }\KeywordTok{dim}\NormalTok{(datos)[}\DecValTok{1}\NormalTok{]}
  \NormalTok{diferencia =}\StringTok{ }\DecValTok{1}

  \NormalTok{while(diferencia >}\StringTok{ }\FloatTok{0.01}\NormalTok{)\{}
  \NormalTok{gradientelog=}\DecValTok{0}
  \NormalTok{sumagradiente=}\DecValTok{0}
  \NormalTok{permutacion =}\StringTok{ }\KeywordTok{sample}\NormalTok{(}\DecValTok{1}\NormalTok{:N,N,}\DataTypeTok{replace=}\OtherTok{FALSE}\NormalTok{)}
  \NormalTok{for(i in }\DecValTok{1}\NormalTok{:N)\{}
    \NormalTok{indice =}\StringTok{ }\NormalTok{permutacion[i]}
    \NormalTok{numerador =}\StringTok{ }\NormalTok{etiqueta[indice]%*%datos[indice,]}
    \NormalTok{denominador =}\StringTok{ }\DecValTok{1}\NormalTok{+}\KeywordTok{exp}\NormalTok{(etiqueta[indice]%*%(datos[indice,]%*%vector_w_ini))}
    \NormalTok{gradientelog =}\StringTok{ }\NormalTok{numerador/denominador[}\DecValTok{1}\NormalTok{,}\DecValTok{1}\NormalTok{]}
    \NormalTok{sumagradiente =}\StringTok{ }\NormalTok{sumagradiente +}\StringTok{ }\NormalTok{gradientelog}
  \NormalTok{\}}
  \NormalTok{vector_w =}\StringTok{ }\NormalTok{vector_w_ini +}\StringTok{ }\NormalTok{tasa*sumagradiente}
  \NormalTok{vector_w =}\StringTok{ }\KeywordTok{as.numeric}\NormalTok{(vector_w)}
  \NormalTok{diferencia =}\StringTok{ }\KeywordTok{sqrt}\NormalTok{(}\KeywordTok{sum}\NormalTok{((vector_w_ini-vector_w)^}\DecValTok{2}\NormalTok{))}
  \NormalTok{vector_w_ini =}\StringTok{ }\NormalTok{vector_w}
  \NormalTok{\}}
  \KeywordTok{return}\NormalTok{(vector_w)}
\NormalTok{\}}
\NormalTok{vector_final =}\StringTok{ }\KeywordTok{reglog}\NormalTok{(}\DataTypeTok{datos=}\NormalTok{datos, }\DataTypeTok{etiqueta=}\NormalTok{etiqueta)}

\CommentTok{#Obtener g}

\CommentTok{#Obtener Eout}
\NormalTok{nuevos_datos =}\StringTok{ }\KeywordTok{simula_unifM} \NormalTok{(N,}\DecValTok{2}\NormalTok{,rango)}
\NormalTok{nuevos_coef =}\StringTok{ }\KeywordTok{simula_recta}\NormalTok{(rango) }
\NormalTok{nueva_f4 =}\StringTok{ }\NormalTok{function(pto,coef) \{          }\CommentTok{#funcion de la recta}
  \NormalTok{pto[}\DecValTok{2}\NormalTok{]-pto[}\DecValTok{1}\NormalTok{]*coef[}\DecValTok{1}\NormalTok{]-coef[}\DecValTok{2}\NormalTok{]}
\NormalTok{\}}
\NormalTok{nuevo_z0 =}\StringTok{ }\KeywordTok{apply}\NormalTok{(nuevos_datos,}\DecValTok{1}\NormalTok{,f4,nuevos_coef)        }\CommentTok{#obtiene los valores de la funcion para datos}
\NormalTok{nueva_etiqueta =}\StringTok{ }\KeywordTok{sign}\NormalTok{(nuevo_z0)                }\CommentTok{# apartir de ellos crea las etiquetas}

\NormalTok{error=}\DecValTok{0}
\NormalTok{resulparcial=}\DecValTok{0}
\NormalTok{for(i in }\DecValTok{1}\NormalTok{:N)\{}
    \NormalTok{resulparcial=}\KeywordTok{log}\NormalTok{(}\DecValTok{1}\NormalTok{+}\KeywordTok{exp}\NormalTok{(-nueva_etiqueta[i]%*%(nuevos_datos[i,]%*%vector_final)))}
    \NormalTok{error =}\StringTok{ }\NormalTok{error +}\StringTok{ }\NormalTok{resulparcial}
\NormalTok{\}}
\NormalTok{error=error/N}
\KeywordTok{print}\NormalTok{(}\StringTok{"Error obtenido:"}\NormalTok{)}
\end{Highlighting}
\end{Shaded}

\begin{verbatim}
## [1] "Error obtenido:"
\end{verbatim}

\begin{Shaded}
\begin{Highlighting}[]
\KeywordTok{print}\NormalTok{(error)}
\end{Highlighting}
\end{Shaded}

\begin{verbatim}
##           [,1]
## [1,] 0.4623543
\end{verbatim}

\end{document}
